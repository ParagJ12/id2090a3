\section{Introduction}
Astrophysical observations of the redshift of stars in galaxies, of the cosmic microwave background anisotropies, of the deflection of the light of distant galaxies, among others, are all explained by the presence of dark matter, which has not shown any interaction with the baryonic matter except the gravitational one.

On the other hand, direct detection of dark matter demands it to have some interaction with the rest of particles of the standard model, see Smith and Lewin (1990) and Gaitskell (2004). These two research lines seem to have contradictory points of view on the properties of the dark matter. In order to attempt to conciliate these seemingly opposite strategies, one can research on the possibility that the astrophysical description of dark matter could relax the pressureless hypothesis, and allow some interaction with itself and/or with the baryonic matter.

\section{Dark matter equation of state from roatational curves}

There are several profiles which have been proposed to adjust the rotational velocities profiles observed in the galaxies eg: Courteau 1997.

We will consider the velocity profile proposed by Persic  obtained by adjusting 1023 galaxies which is given by
\begin{equation}
	\beta^2 =\beta_0(\,(x^2)/(x^2+a^2))\,
\end{equation}$$$$

where a and $\beta_0$ are constants. The first one, $\beta_0$, stands for the ratio of the terminal velocity to the speed of light, and a determines how fast the velocity reaches a terminal value. 

$\beta_0$ is a function of the luminosity of the galaxy. In our work, we will assume that both $\beta_0$ and a are simply constants that can be fitted to observational data and from now we will refer to equation (1) as the PSS rotational velocity profile.
Given $\beta_2$(x) by equation (1), it is straightforward to obtain from equations (2)–(4) analytical expressions for the mass function, the density, and the pressure

 \begin{equation}
         n=((\beta\underset{0})^2/q)[x^3/x^2+a^2]
 \end{equation}

\begin{equation}
    \bar{\rho}=((\beta\underset{0})^2/3q)[(x^2+3a^2)/(x^2+a^2)^2]
\end{equation}

\begin{equation}
    \bar{p}=(\beta\underset{0})^4/6q[(x^2+2a^2)/(x^2+a^2)^2]
\end{equation}


\section{student details}
\begin{enumerate}
    \item  Name:Shri Dharshan
    \item Roll NO:ME22B043
    \item github profile: shridharshan69
\end{enumerate}
 
\footnote{Monthly Notices of the Royal Astronomical Society, Volume 449, Issue 1, 1 May 2015, Pages 403–413, https://doi.org/10.1093/mnras/stv302

published on:17 march 2015}

