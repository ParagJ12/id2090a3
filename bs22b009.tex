\documentclass[12pt,a4paper]{article}
\usepackage{amsmath}
\usepackage{amsfonts}
\usepackage{amssymb}
\usepackage{graphicx}
\usepackage[left=2cm,right=2cm,top=2cm,bottom=2cm]{geometry}

\author{Aryan Shirbhate}
\title{Speed of Wave}

\date{June 07, 2022}

\begin{document}

\maketitle

    \maketitle 
    
    \section{Relation Between Velocity And Wavelength} 
    
    \textbf Wavelength is the measure of the length of a complete wave cycle. The velocity of a wave is the distance travelled by a point on the wave. In general, for any wave, the relation between velocity and wavelength is proportionate. It is expressed through the wave velocity formula. 


    
    \begin{equation}
    \text{v} = \mu \cdot \lambda
    \label{eqn:v}
\end{equation}




    Where,
\begin{enumerate}
    \item V is the velocity of the wave measured using m/s.
    
    \item $\mu$ is the frequency of the wave measured using Hz.
    
    
    \item $\lambda$ is the wavelength of the wave measured using m.

\end{enumerate} 

\section{Velocity and Wavelength Relation}

Amplitude, frequency, wavelength, and velocity are the characteristic of a wave. For a constant frequency, the wavelength is directly proportional to velocity.

Given by:

V  $\propto$ $\lambda$

Example:

\begin{itemize}
    \item For a constant frequency, if the wavelength is doubled, the velocity of the wave will also double.
    \item For a constant frequency, if the wavelength is made four times, the velocity of the wave will also be increased by four times.
\end{itemize}


    
    
     
\footnote{https://byjus.com/physics/relation-between-velocity-and-wavelength/}
\footnote{Github-id: Aryan7865}
\end{document}
